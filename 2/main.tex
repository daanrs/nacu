\documentclass{scrartcl}
\usepackage[utf8]{inputenc}
\usepackage{pgf-pie}
\usepackage{subfigure}
\usepackage{hyperref}
\usepackage{cite}
\usepackage{amsmath}
\usepackage{amssymb}
\usepackage{graphicx}
%\documentclass{book}
\renewcommand\thesubsection{\Alph{subsection}}


%opening
\title{Assignment 1}
\author{Daan Spijkers, s1011382\\ Tomás Catalán López, s1081589\\ Willem Lambooy, s1009584}

\begin{document}
\maketitle

\textbf{Note: you can find the GitHub repository at}
\url{github.com/dspkio/nacu}.

\subsection*{4}

\begin{itemize}

  \item[(a)]
    No, it is not. Our space of valid solutions is $\{\{e_1, e_2\}, \{e_2,
    e_3\}, \{e_3, e_4\}\}$. We see that $e_1$ and $e_4$ occur only once,
    and $e_2$, $e_3$ occur twice.

  \item[(b)]
    Not always. Our goal is finding the optimal solution as efficiently as
    possible, if a bias is helpful for that then it is not harmful. If,
    for example, in the example given in (a) the weights for $e_2$ and
    $e_3$ were \emph{less} than the weights for $e_1$ and $e_4$, then this
    would be a positive thing.

\end{itemize}

\subsection*{5}
The intuitive understanding is that ants going up will get a path straight
to the destination, while ants going down will get stuck, and have longer
paths. Even though the shortest path (length $5$) is down, it is possible
that the ants will converge on the upper path (length $8$).

If we calculate the expected length of the paths, then we can use that as
an indication for what the ant colony could converge on (the shorter the
expected length, the more pheromone). For the up path,
it is obvious that $E(up) = 9$.

For the down path, we can provide a lower bound by enumerating the
probabilities for all paths shorter than 9, and leaving all other paths
with length \emph{at least 10}.
\begin{equation}
  6  P(6) + 7  P(7) + 8  P(8) + 9  P(9) + 10(1 - P(\le9)) \le E(down)
\end{equation}

In the next section we assume that we initially go down. Describing the
enumeration is a bit cumbersome, but please bear with us. For the shortest
path of length 6 we need to choose the correct middle one at the 3rd
vertex (out of 3 choices), and then choose correctly again at the 4th and
5th vertex. This gives it $P(6) = \frac{1}{3} \times \frac{1}{2} \times
\frac{1}{2} = \frac{1}{12}$.

Length 7 is the exact same. For length 8 we can have a similar one of
probability $\frac{1}{12}$, but another way is to have the same initial
path of length 6, but then "miss" the 4th one, and then choose
correctly twice after that. This gives it $P(8) = \frac{1}{12} +
\frac{1}{48}$. Similarly, $P(9) = \frac{1}{48}$, as we can only get it by
"missing" once in the 7 length path. Now, $P(>9) = 1 - \frac{2}{12} -
\frac{2}{48} = \frac{19}{24}$, and the lower bound is:

\begin{equation}
  10 = \frac{6}{12} + \frac{7}{12} + 8(\frac{1}{12} + \frac{1}{48}) +
  \frac{9}{48} + 10 \times \frac{19}{24} \le E(down)
\end{equation}

So, the expected length for paths going down is at least $1$ higher
than the expected length for paths going up. This means it is quite
possible for things to converge on the up path. This is still random,
obviously, and so it will depend on the randomness and exact parameters of
the algorithm which path it will actually find.

\end{document}
